% Copyright © 2015 Fabian Köhler <fkoehler1024@googlemail.com>
\usetikzlibrary{math}
\usetikzlibrary{arrows.meta}

% TODO: rename procX1 -> blockX1, ...
% TODO: unused val for unuesed parameters
% TODO: FZJ corporate design
% TODO: space for extra highlighting ?
% TODO: brace and label for highlitght function

%%%%%%%%%%%%%%%%%
% Configuration %
%%%%%%%%%%%%%%%%%
% config variables
\tikzmath{
  % number of processors in the diagram (not used in manual block drawing)
  \numProc = 4;
  %
  % width of sweep blocks
  \blockWidth = 1.8;
  %
  % horizontal space between blocks
  \blockHSpace = 0.9;
  %
  % vertical space between blocks
  \blockVSpace = 0.08;
  %
  % height of coarse sweep blocks
  \coarseHeight = 0.4;
  %
  % height of fine sweep blocks
  \fineHeight = 1.3;
  %
  % spacing between y-axis and pfasst drawing
  \axisHSpace = 1.001*\blockHSpace;
  %
  % spacing between x-axis and pfasst drawing
  \axisVSpace = 0.15;
  %
  % length of ticks
  \ticksLength = 0.1;
  %
  % offset for ticks labels
  \ticksOffset = 0.4;
  %
  % if drawComm == 1 the arrows for communication will be drawn
  \drawComm = 1;
  %
  % ratio at which the triangle for coarse comm arrows will be drawn
  \coarseCommRatio = 0.4;
  %
  % ratio at which the triangle for fine comm arrows will be drawn
  \fineCommRatio = 0.8;
  %
  % radius for the small circles at the end of comm arrows
  \commCircleRadius = 0.03;
}

%%%%%%%%%%%%%%%%%%%%%%%%%%%%%%%%%%%%%%%%%%%%%%%%%%%%
% draw functions for elements (blocks, lines, ...) %
%%%%%%%%%%%%%%%%%%%%%%%%%%%%%%%%%%%%%%%%%%%%%%%%%%%%
\tikzmath{
  % specify how coarse sweep blocks will be drawn
  function drawCoarseSweep(\xa, \ya, \xb, \yb) {
      { \filldraw[draw=black,fill=red] (\xa, \ya) rectangle (\xb, \yb); };
  };
  %
  % specify how fine sweep blocks will be drawn
  function drawFineSweep(\xa, \ya, \xb, \yb) {
      { \filldraw[draw=black,fill=blue] (\xa, \ya) rectangle (\xb, \yb); };
  };
  %
  % specify how axes are drawn
  function drawAxis(\xa, \ya, \xb, \yb) {
      { \draw[->,>=latex,thick] (\xa, \ya) -- (\xb, \yb); };
  };
  %
  % draw a label tick for a processor
  function drawProcTick(\x, \proc) {
      { \draw (\x, -\axisVSpace) -- (\x, -\axisVSpace-\ticksLength); };
      { \node at (\x,-\ticksLength-\ticksOffset) {$P_{\pgfmathprintnumber{\proc}}$}; };
  };
  %
  % draw a labeled tick for a time step
  function drawTimeTick(\x, \i) {
      { \draw (\x, -\axisVSpace) -- (\x, -\axisVSpace-\ticksLength); };
      { \node[text=black!60!white] at (\x,-\ticksLength-\ticksOffset) {$t_{\pgfmathprintnumber{\i}}$}; };
  };
  %
  % draw coarse communication
  function drawCoarseComm(\xa, \ya, \xb, \yb) {
      \xh = (\coarseCommRatio*\xb+(1-\coarseCommRatio)*\xa);
      \yh = (\coarseCommRatio*\yb+(1-\coarseCommRatio)*\ya);
      { \draw[thick,-{Triangle[open]}] (\xa,\ya) -- (\xh,\yh); };
      { \draw[thick] (\xh,\yh) -- (\xb,\yb); };
      { \fill[draw=black,fill=white] (\xa,\ya) circle[radius=\commCircleRadius]; };
      { \fill[draw=black,fill=white] (\xb,\yb) circle[radius=\commCircleRadius]; };
  };
  %
  % draw fine communication
  function drawFineComm(\xa, \ya, \xb, \yb) {
      \xh = (\fineCommRatio*\xb+(1-\fineCommRatio)*\xa);
      \yh = (\fineCommRatio*\yb+(1-\fineCommRatio)*\ya);
      { \draw[thick, -{Triangle[open]},>=latex,black!60!green] (\xa,\ya) -- (\xh,\yh); };
      { \draw[thick, black!60!green] (\xh,\yh) -- (\xb,\yb); };
      { \fill[draw=black!60!green,fill=white] (\xa,\ya) circle[radius=\commCircleRadius]; };
      { \fill[draw=black!60!green,fill=white] (\xb,\yb) circle[radius=\commCircleRadius]; };
  };
  %
  % label the y axis
  function labelYAxis(\pfasstSteps) {
      \y = getPfasstHeight(-\ticksOffset, \pfasstSteps)/2;
      { \node[rotate=90] at (-\axisHSpace-\ticksOffset, \y) { computation time }; };
  };
  %
  %
  function drawHightlightLine(\xa, \ya, \xb, \yb) {
      { \draw (\xa,\ya) -- (\xb,\yb); };
  };
}



%%%%%%%%%%%%%%%%%%%%%%%%%%%%%%%%%%%%%
% Functions to calculate properties %
%%%%%%%%%%%%%%%%%%%%%%%%%%%%%%%%%%%%%
\tikzmath{
  % get the x coordinate where a process starts
  function procX1(\proc) {
      if \proc == 0 then {
        return \proc*\blockWidth;
      } else {
        return \proc*\blockWidth+\proc*\blockHSpace;
      };
  };
  %
  % get the x coordinate where a process ends
  function procX2(\proc) {
      if \proc == 0 then {
          return (\proc+1)*\blockWidth;
      } else {
          return (\proc+1)*\blockWidth+\proc*\blockHSpace;
      };
  };
  %
  % get the y coordinate where to start a block
  % the number of coarse and fine blocks in the stack has to be specified
  function blockY(\nCoarse, \nFine) {
      return \nCoarse*\coarseHeight+\nFine*\fineHeight+(\nCoarse+\nFine)*\blockVSpace;
  };
  %
  % number of coarse sweeps in prediction on a processor
  function numPredictBlocks(\proc) {
      return \proc+1;
  };
  %
  % number of coarse sweeps in prediction up to a certain step on a processor
  function numPredictBlocksStep(\proc, \step) {
      if \step >= \proc then {
        return \proc+1;
      } else {
        return \step+1;
      };
  };
  %
  % calculate the width of the pfasst drawing according to the number of processors
  function getPfasstWidth(\unused) {
    return (\numProc-1)*(\blockWidth+\blockHSpace)+\blockWidth;
  };
  %
  % calculate the height of the pfasst drawing according to the number of pfasst steps
  function getPfasstHeight(\pfasstSteps) {
      \predictHeight = (\numProc-1)*(\coarseHeight+\blockVSpace)+\coarseHeight;
      if \pfasstSteps == 0 then {
          return \predictHeight;
      };
      \h = (\pfasstSteps-1)*(\coarseHeight+\fineHeight+2*\blockVSpace)+\coarseHeight+\fineHeight+\blockVSpace;
      return \h+\predictHeight;
  };
  %
  % get the x position of the tick for a certain processor
  function getProcTickX(\proc) {
      if \proc == 0 then {
          return \blockWidth/2;
      } else {
          return \proc*(\blockWidth+\blockHSpace)+\blockWidth/2;
      };
  };
  %
  % get the x position of the tick for a certain time step
  function getTimeTickX(\i) {
      if \i == 0 then {
          return -\blockHSpace/2;
      } else {
          return (\i-1)*(\blockWidth+\blockHSpace)+\blockWidth+\blockHSpace/2;
      };
  };
}


%%%%%%%%%%%%%%%%%%%%%%%%%%%%%%%%%%%%%%%%%%%%%
% Functions to draw individual sweep blocks %
%%%%%%%%%%%%%%%%%%%%%%%%%%%%%%%%%%%%%%%%%%%%%
\tikzmath{
  % draw a coarse sweep block on stack of a certain processor that contains
  % a given number of coarse and fine blocks
  function coarseSweep(\proc, \nCoarse, \nFine) {
      \xa = procX1(\proc);
      \xb = procX2(\proc);
      \ya = blockY(\nCoarse, \nFine);
      \yb = \ya+\coarseHeight;
      drawCoarseSweep(\xa, \ya, \xb, \yb);
      if \drawComm == 1 then {
          if \proc > 0 then {
              if \nCoarse > 0 then {
                  \xa = procX2(\proc-1);
                  \ya = blockY(\nCoarse-1, \nFine)+\coarseHeight;
                  \xb = procX1(\proc);
                  \yb = blockY(\nCoarse, \nFine);
                  drawCoarseComm(\xa,\ya,\xb,\yb);
              };
          };
      };
  };
  %
  % draw a fine sweep block on stack of a certain processor that contains
  % a given number of coarse and fine blocks
  function fineSweep(\proc, \nCoarse, \nFine) {
      \xa = procX1(\proc);
      \xb = procX2(\proc);
      \ya = blockY(\nCoarse, \nFine);
      \yb = \ya+\fineHeight;
      drawFineSweep(\xa, \ya, \xb, \yb);
      if \drawComm == 1 then {
          if \proc > 0 then {
              if \nFine > 0 then {
                  \xa = procX2(\proc-1);
                  \ya = blockY(\nCoarse-2, \nFine-1)+\fineHeight;
                  \xb = procX1(\proc);
                  \yb = blockY(\nCoarse, \nFine);
                  drawFineComm(\xa, \ya, \xb, \yb);
              };
          };
      };
  };
}



%%%%%%%%%%%%%%%%%%%%%%%%%%%%%%%%%%%%%%%%%%%%%%%%%%%%
% Functions to draw bigger parts of pfasst at once %
%%%%%%%%%%%%%%%%%%%%%%%%%%%%%%%%%%%%%%%%%%%%%%%%%%%%
\tikzmath{
  % draw a single step of the prediction on all processors
  function predictStep(\step) {
      for \proc in {0, ..., \numProc-1}{
          \n = numPredictBlocksStep(\proc, \step);
          if \step+1 <= \n then {
              coarseSweep(\proc, \n-1, 0);
          };
      };
  };
  %
  % draw fine sweeps of pfasst step
  function pfasstStepFine(\step) {
      for \proc in {0, ..., \numProc-1}{
          \nCoarse = \step+numPredictBlocks(\proc);
          \nFine   = \step;
          fineSweep(\proc, \nCoarse, \nFine);
      };
  };
  %
  % draw coarse sweeps of pfasst step
  function pfasstStepCoarse(\step) {
      for \proc in {0, ..., \numProc-1}{
          \nCoarse = \step+numPredictBlocks(\proc);
          \nFine   = \step+1;
          coarseSweep(\proc, \nCoarse, \nFine);
      };
  };
  %
  % draw the full prediction on all processors
  % the parameter is not used, but it seems that you cannot use functions
  % without parameters
  function predict(\unused) {
      for \step in {0, ..., \numProc-1}{
          predictStep(\step);
      };
  };
  %
  % draw a full pfasst step
  function pfasstStep(\step) {
      pfasstStepFine(\step);
      pfasstStepCoarse(\step);
  };
  %
  % draw the full algorithm with a specified number of pfasst steps
  function pfasst(\steps) {
      predict(0);
      for \step in {0, ..., \steps-1}{
          pfasstStep(\step);
      };
  };
}



%%%%%%%%%%%%%%%%%%%%%%%%%%%%%%%%%%%%%%%%%%%%%%%%%
% Functions to highlight parts of the algorithm %
%%%%%%%%%%%%%%%%%%%%%%%%%%%%%%%%%%%%%%%%%%%%%%%%%
\tikzmath{
  function highlightPredictor(\unused) {
      for \proc in {0, ..., \numProc-1}{
          % horizontal line
          \xa = procX1(\proc);
          \xb = procX2(\proc);
          \ya = blockY(numPredictBlocks(\proc-1), 0)+\coarseHeight+\blockVSpace/2;
          \yb = \ya;
          drawHightlightLine(\xa, \ya, \xb, \yb);
          if \proc != \numProc-1 then {
              % diagonal line
              \xa = procX2(\proc);
              \xb = procX1(\proc+1);
              \yb = \yb+\coarseHeight+\blockVSpace;
              drawHightlightLine(\xa, \ya, \xb, \yb);
          };
      };
  };
  function highlightPfasstStep(\step) {
      for \proc in {0, ..., \numProc-1}{
          % horizontal line
          \nCoarse = numPredictBlocks(\proc);
          \nFine = 0;
          if \step > 0 then {
              \nCoarse = \nCoarse+\step+1;
              \nFine = \step+1;
          };
          \xa = procX1(\proc);
          \xb = procX2(\proc);
          \ya = blockY(\nCoarse+1, \nFine)+\fineHeight+\blockVSpace/2;
          \yb = \ya;
          drawHightlightLine(\xa, \ya, \xb, \yb);
          if \proc != \numProc-1 then {
              % diagonal line
              \xa = procX2(\proc);
              \xb = procX1(\proc+1);
              \yb = \yb+\coarseHeight+\blockVSpace;
              drawHightlightLine(\xa, \ya, \xb, \yb);
          };
      };
  };
}



%%%%%%%%%%%%%%%%%%%%%%%%%%
% Functions to draw axes %
%%%%%%%%%%%%%%%%%%%%%%%%%%
\tikzmath{
  % draw the x-axis
  function drawXAxis(\unused) {
      \w = getPfasstWidth(0)+\axisHSpace;
      drawAxis(-\axisHSpace, -\axisVSpace, \w, -\axisVSpace);
      for \proc in {0, ..., \numProc-1}{
          \x = getProcTickX(\proc);
          drawProcTick(\x, \proc);
          \x = getTimeTickX(\proc);
          drawTimeTick(\x, \proc);
      };
      \x = getTimeTickX(\numProc);
      drawTimeTick(\x, \numProc);
  };
  % draw the y axis according to the number of pfasst steps
  function drawYAxis(\pfasstSteps) {
      \h = getPfasstHeight(\pfasstSteps)+\axisVSpace;
      drawAxis(-\axisHSpace, -\axisVSpace, -\axisHSpace, \h);
      labelYAxis(\pfasstSteps);
  };
  % draw axes according to the number of pfasst steps and number of processors
  function drawAxes(\pfasstSteps) {
      drawXAxis(0);
      drawYAxis(\pfasstSteps);
  };
}
